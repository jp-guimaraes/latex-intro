%----------------------------------------------------------------------------------------
%	PACKAGES AND THEMES
%----------------------------------------------------------------------------------------
\documentclass[aspectratio=169,xcolor=dvipsnames]{beamer}
\usetheme{Simple}

\usepackage{hyperref}
\usepackage{graphicx} % Allows including images
\usepackage{booktabs} % Allows the use of \toprule, \midrule and \bottomrule in tables
\usepackage{listings}
\usepackage{color}


%----------------------------------------------------------------------------------------
%	TITLE PAGE
%----------------------------------------------------------------------------------------

% The title
\title[Escrita com latex]{Introdução à escrita científica com \LaTeX}
\subtitle{Expotec 2021}

\author[João Guimarães] {João P. F. Guimarães}
\institute[IFRN]{
    \href{https://jp-guimaraes.github.io/}{https://jp-guimaraes.github.io/} \\ 
    \vskip 10pt
    IFRN - Campus João Câmara
    \vskip 3pt
}
\date{\today} % Date, can be changed to a custom date


%----------------------------------------------------------------------------------------
%	PRESENTATION SLIDES
%----------------------------------------------------------------------------------------

\begin{document}

\begin{frame}
    % Print the title page as the first slide
    \titlepage
\end{frame}



\begin{frame}{Disclaimer}
    \begin{alertblock}{Alerta!}
        Workshop contaminado pelo viés do autor
    \end{alertblock}
    \begin{itemize}
        \item Experiência relacionada à pesquisa em Engenharia Elétrica/Computação
    \end{itemize}
\end{frame}

\begin{frame}{Sumário}
    \tableofcontents
\end{frame}

\section{Escrita científica}

\begin{frame}{Textos científicos}
    \begin{itemize}
        \item Monografias
        \item Dissertações
        \item Teses
        \item Artigos científicos
        \item Livros
    \end{itemize}
\end{frame}


\begin{frame}{Especificidades de um artigo científico}

    \begin{itemize}
        \item Segue/suporta o método científico
        \item Seções que dividem a exposição das ideias: \\
        Título/Abstract/Introdução/Referencial teórico/Metodologia/Resultados/Conclusões \\
        Título, Resumo, Introdução, Materiais e Métodos, Resultados, Discussão, Conclusão, Agradecimentos e Referências
        \item Uso de referências para suporte de afirmações (Bibliografia)
        \item Tabelas, figuras, equações, algoritmos, etc
    \end{itemize}
    
\end{frame}

\begin{frame}{Artigos científicos}

    \begin{itemize}
        \item Texto científico
        \item Principal forma de comunicação e divulgação científica
        \item Cientista $\rightarrow$ Sociedade (Comunidade científica)
    \end{itemize}

\end{frame}

\begin{frame}{Periódicos e Congressos}
    \begin{itemize}
        \item Um dos principais meios de divulgação científica
        \item Cada periódico(revista)/congresso tem o seu escopo e metodologia
        \item Cada revista tem o seu padrão de formatação
    \end{itemize}
\end{frame}
    
\begin{frame}{Tipos de formatação utilizadas}
    \begin{itemize}
        \item Coluna simples/coluna dupla
        \item Alinhamento
        \item Modelo de referência
        \item Modelo de citação
        \item Tipo de título
        \item Autores 
        \item Rodapé
        \item Seções
        \item Numeração de seções
        \item Biografia dos autores 
        \item Estilo de bibliografia
    \end{itemize}
    Vamos analisar alguns exemplos!
\end{frame}

\section{O que é \LaTeX?}

\begin{frame}{O que é \LaTeX?}
    \begin{itemize}
        \item Sistema de preparação de documentos 
        \item Lançado em 1983
        \item Foco no conteúdo e não na formatação. 
        \item Texto simples vs WYSIWYG  
    \end{itemize}
\end{frame}

\begin{frame}{Vantagens do uso do \LaTeX}
    \begin{itemize}
        \item Foco principal no conteúdo do texto
        \item Arquivos de estilo que modificam os padrões de formatação conforme periódico escolhido
        \item Reaproveitamento de texto e arquivo de referências 
    \end{itemize}
        
\end{frame}

\begin{frame}{Fluxo de trabalho no \LaTeX}
    \begin{itemize}
        \item Necessário instalar (Windows, Linux, Mac)
        \item Escreve arquivo .tex 
        \item Necessário compilar .tex para gerar .ps .pdf, etc
    \end{itemize}
\end{frame}

\begin{frame}{Formas de trabalho}
    \begin{itemize}
        \item Uso de IDEs: \href{https://beebom.com/best-latex-editors/}{alguns exemplos}
        \item "No braço" (editor de texto mais compilação via terminal)
        \item Ferramentas online - \href{http://www.overleaf.com/}{Overleaf}
    \end{itemize}
\end{frame}

\begin{frame}{Estrutura de um projeto \LaTeX }
    \begin{itemize}
        \item Arquivos fonte (texto, imagens, bibliografia)
        \item Arquivos auxiliares (estilo, comandos, configurações)
        \item Arquivos finais (pdf, jpg, ps)
    \end{itemize}
\end{frame}

\section{Overleaf}

\begin{frame}{Overleaf}
    \begin{itemize}
        \item IDE online
        \item Multiplataforma
        \item Nuvem
        \item Trabalho em equipe
    \end{itemize}
    Overleaf vai ser a solução usada nesse Workshop
\end{frame}



\begin{frame}{Estrutura de um projeto \LaTeX no Overleaf }
     \begin{itemize}
        \item Menu
        \item Arquivos usados
        \item Editor de texto
        \item Visualizador do PDF
        \item Recursos da Nuvem
    \end{itemize}
\end{frame}

\section{Exemplos}

\begin{frame}{Anatomia de um arquivo \LaTeX}
    \begin{itemize}
        \item Classe do documento 
        \item Preâmbulo (temas, pacotes, metadados)
        \item Texto
        \item Bibliografia (pode estar num arquivo separado)
    \end{itemize}
\end{frame}

\begin{frame}
    \Huge{\centerline{Vamos aos exemplos!}}
\end{frame}

\begin{frame}{Roteiro}
    \begin{itemize}
        \item Arquivo simples (teste de alguns recursos)
        \item Modelo Exemplo - Expotec (Galeria do Overleaf)
        \item Análise do Template de TCC - IFRN/JC
    \end{itemize}
\end{frame}

\begin{frame}{Dicas}

    \href{https://pt.overleaf.com/learn}{Documentação do Overleaf em inglês}
    
    
    \href{https://en.wikibooks.org/wiki/LaTeX/Document_Structure}{Wikibook com informações gerais sobre \LaTeX em inglês}
    
    \href{https://wp.ufpel.edu.br/nuclear/files/2017/04/intlatex.pdf}{Ótima introdução ao \LaTeX em português}

    
\end{frame}

\begin{frame}
    \Huge{\centerline{Obrigado!}} 
\end{frame}


\end{document}